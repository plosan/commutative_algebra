\usepackage[T1]{fontenc}
\usepackage[english]{babel}
\usepackage{csquotes}
\usepackage{enumitem}

\usepackage[a4paper, left = 30mm, right = 20mm, top = 30mm, bottom = 20mm]{geometry}

\usepackage{amsthm}
\usepackage{amsmath}
\usepackage{amssymb}
\usepackage{amsfonts}
\usepackage{mathtools}
\usepackage{mathrsfs}
\usepackage{empheq}
\usepackage{physics}


\usepackage[labelsep=period, labelfont=bf]{caption}

\usepackage{fancyhdr}


\usepackage{titling}
\usepackage{placeins} % FloatBarrier

\usepackage{tikz}
\usepackage{tikz-cd}
\usetikzlibrary{arrows}
\usetikzlibrary{decorations.markings}

%\usepackage{eulervm}
%\usepackage{tgtermes}
\usepackage{lmodern}
%\usepackage{mathptmx}

\usepackage{hyperref}
\hypersetup{
	colorlinks,
	citecolor=black,
	filecolor=black,
	linkcolor=blue,
	urlcolor=black
}


% Estilo de página
\pagestyle{fancy}
\fancyhf{}
\setlength{\headheight}{15pt}
\fancyhead[LE,RO]{\leftmark}
\fancyhead[RE,LO]{\thepage}
\setlength{\parskip}{0.25cm}

% General commands
\newcommand{\emphdef}[1]{\textbf{#1}}

% Ring commands
\newcommand{\spec}[1]{\mathrm{Spec}\,#1}
\newcommand{\spm}[1]{\mathrm{Spm}\,#1}
\newcommand{\maxideals}[1]{\mathrm{Max}\,#1}
\newcommand{\rad}[1]{\mathrm{rad}(#1)}
\newcommand{\ann}[2]{\mathrm{Ann}_{#1}(#2)}

% Math commands
\newcommand{\n}{\mathbb{N}}
\newcommand{\z}{\mathbb{Z}}
\newcommand{\q}{\mathbb{Q}}
\renewcommand{\real}{\mathbb{R}}
\newcommand{\cplex}{\mathbb{C}}
\renewcommand{\k}{\mathbb{K}}
\newcommand{\f}{\mathbb{F}}

\newcommand{\matrices}[2]{\mathcal{M}_{#1}(#2)}

\newcommand{\car}[1]{\mathrm{char}\left(#1\right)}
\newcommand{\class}[1]{\overline{#1}}
\newcommand{\mcd}[1]{\mathrm{mcd}{#1}}
\renewcommand{\mod}[1]{\ (\mathrm{mod}\ #1)}
\newcommand{\fractions}[1]{\mathrm{Fr}(#1)}
\renewcommand{\gcd}[1]{\mathrm{mcd} #1}
\newcommand{\mcm}[1]{\mathrm{mcm} #1}
\newcommand{\conj}[1]{\overline{#1}}
\newcommand{\degext}[2]{\left[ #1 \, : \, #2 \right]}
\newcommand{\irr}{\mathrm{Irr}}
\newcommand{\rg}[1]{\mathrm{rg}\left(#1\right)}
\newcommand{\divides}{\mid}
\newcommand{\notdivides}{\nmid}
\newcommand{\scr}[1]{\left\langle #1 \right\rangle}

% Comandos grupos
\newcommand{\generated}[1]{\left\langle #1 \right\rangle}
\newcommand{\symgroup}[1]{\mathcal{S}_{#1}}
\newcommand{\indexgroup}[2]{\left[ #1 \, : \, #2 \right]}

% Conjunto vacío
\let\oldemptyset\emptyset
\let\emptyset\varnothing

% Imagen
\DeclareMathOperator{\image}{Im}

% Abreviaciones
\newcommand*{\eg}{e.g.\@\xspace}
\newcommand*{\ie}{i.e.\@\xspace}

% Conjunto cociente
\newcommand{\quotient}[2]{{\raisebox{.2em}{$#1$}\left/\raisebox{-.2em}{$#2$}\right.}}

% Eliminar Capitulo, solo número y nombre
\makeatletter
\def\@makechapterhead#1{%
	\vspace*{50\p@}%
	{\parindent \z@ \raggedright \normalfont
		\ifnum \c@secnumdepth >\m@ne
		\if@mainmatter
		%\huge\bfseries \@chapapp\space \thechapter
		\huge\bfseries \thechapter.\space%
		%\par\nobreak
		%\vskip 20\p@
		\fi
		\fi
		\interlinepenalty\@M
		\huge \bfseries #1\par\nobreak
		\vskip 40\p@
}}
\makeatother

% Numerales romanos en el texto
\makeatletter
\newcommand*{\rom}[1]{\expandafter\@slowromancap\romannumeral #1@}
\makeatother

% cdot mas grande para productos en grupos
\makeatletter
\newcommand*\bcdot{\mathpalette\bcdot@{.5}}
\newcommand*\bcdot@[2]{\mathbin{\vcenter{\hbox{\scalebox{#2}{$\m@th#1\bullet$}}}}}
\makeatother

% Ceil
\DeclarePairedDelimiter\ceil{\lceil}{\rceil}
% Floor
\DeclarePairedDelimiter\floor{\lfloor}{\rfloor}

% Espacio antes y despues de ":"
\DeclareMathSymbol{:}{\mathord}{operators}{"3A}

% % Swap de phi y varphi
%\let\temp\phi
%\let\phi\varphi
%\let\varphi\temp




