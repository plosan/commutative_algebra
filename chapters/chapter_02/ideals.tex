
\section{Ideals}

\begin{definition}
	Let $(R,+,\vdot)$ be a ring (not necessarily commutative nor with unit).
	\begin{enumeratedef}
		\item A subset $I \subset R$ is a \emphdef{left ideal} of $R$ if
		\begin{enumerate}[label={(i.\alph*)}]
			\item $(I,+)$ is an abelian group
			\item For every $a \in R$ and $r \in R$, $r a \in I$.
		\end{enumerate}
		\item A subset $I \subset R$ is a \emphdef{right ideal} of $R$ if 
		\begin{enumerate}[label={(ii.\alph*)}]
			\item $(I,+)$ is an abelian group
			\item For every $a \in R$ and $r \in R$, $a r \in I$.
		\end{enumerate}
		\item A subset $I \subset R$ that is both a left ideal and a right ideal, is called an \emphdef{ideal} of $R$.
	\end{enumeratedef}
\end{definition}

If $R$ is a commutative ring, then left and right ideals coincide and are simply called ideals. Moreover, if $R$ is a commutative ring with unit and an ideal $I \subset R$ contains the unit, $1_R \in I$, then $I = R$.

\begin{definition}
	Let $R$ be a commutative ring with unit. A family $\{ f_{\lambda} \}_{\lambda \in \Lambda}$ of elements of $R$ is a \emphdef{system of generators of an ideal $I \subset A$} if every element $f \in I$ can be expressed as a finite linear combination
	\[
		f = a_1 f_{\lambda_1} + \cdots + a_n f_{\lambda_r}, \quad a_1, \ldots, a_r \in R
	\]
	In this case we write $I = (f_{\lambda} \mid \lambda \in \Lambda)$ to denote that $I$ is generated by the $f_\lambda$. If the family $\{ f_1, \ldots, f_r \}$ is finite, then $I = (f_1, \ldots, f_r)$ and we say that $I$ is \emphdef{finitely generated}. If $I$ is generated by a single element, that is, $I = (f) = \{ r f \mid r \in R \}$, we say that $I$ is a \emphdef{principal ideal}.
\end{definition}

Hereinafter $R$ will denote a commutative ring with unit. In the coming sections we shall study some manners to construct new ideals from the given ones.

\subsection{Intersection of ideals}

\begin{prop}
	Let $R$ be a ring.
	\begin{enumerateprop}
		\item If $I, J \subset R$ are ideals, then the intersection $I \cap J$ is an ideal of $R$.
		\item Given any family $\{ I_\lambda \}_{\lambda \in \Lambda}$ of ideals of $R$, the intersection $\bigcap_{\lambda \in \Lambda} I_\lambda \subset R$ is an ideal of $R$.
	\end{enumerateprop}
\end{prop}

\subsection{Sum of ideals}

\begin{definition}
	Let $R$ be a ring and $\{ I_\lambda \}_{\lambda \in \Lambda}$ an arbitary family of ideals of $R$. We define the \emphdef{sum ideal} by
	\[
		\sum_{\lambda \in \Lambda} I_\lambda = 
		\left\{ 
			\sum\nolimits_{\lambda \in \Lambda} x_\lambda
			\left\lvert
			x_\lambda \in I_\lambda, \ x_\lambda = 0 \text{ except for finitely many } \lambda \in \Lambda
			\right.
		\right\}
	\]
	If the family $\{ I_1, \ldots, I_r \}$ of ideals of $R$ is finite, the sum ideal is simply
	\[
		\sum_{\lambda = 1}^r I_k = 
		\left\{ 
		\sum\nolimits_{\lambda = 1}^r x_\lambda
		\left\lvert
		x_\lambda \in I_\lambda
		\right.
		\right\}
	\]
\end{definition}

\begin{prop}
	Let $R$ be a ring. Then the sum ideal of an arbitrary family $\{ I_\lambda \}_{\lambda \in \Lambda}$ of ideals of $R$ is again an ideal of $R$.
\end{prop}


\subsection{Product of ideals}

\begin{definition}
	Let $R$ be a ring and let $I, J$ be ideals of $R$. We define the \emphdef{product ideal} by
	\[
		I J = 
		\left\{
			\sum_{\lambda = 1}^n x_\lambda y_\lambda \mid
			x_\lambda \in I, \, y_\lambda \in J, \, n > 0
		\right\}
	\]
	If $\{ I_1, \ldots, I_r \}$ is a finite family of ideals of $R$, the product ideal is 
	\[
		I_1 \cdots I_r = 
		\left\{
			\sum_{\lambda = 1}^n x_{\lambda,1} \cdots x_{\lambda,r} \mid
			x_{\lambda,i} \in I_i, \, n > 0
		\right\}
	\]
\end{definition}

\begin{prop}
	Let $R$ be a ring and let $\{ I_1, \ldots, I_r \}$ be a finite family of ideals of $R$. Then their product ideal $I_1 \cdots I_r$ is an ideal of $R$. Moreover, it is the ideal generated by the set
	\[
		S = \{ x_1 \cdots x_r \mid x_i \in I_i \}
	\]
\end{prop}

\subsection{Ideal generated by a subset}

In general, an arbitrary subset $S \subset R$ will not be an ideal. Nonetheless, in some cases we shall need the smallest ideal, in the sense of inclusions, that contains $S$. 

\begin{definition}
	Let $R$ be a commutative ring with unit and $S \subset R$ a subset. The \emphdef{ideal generated by $S$} is
	\[
	I = \bigcap_{\substack{J \subset A \text{ ideal} \\ S \subset J}} J
	\]
\end{definition}

\begin{prop}
	Let $R$ be a commutative ring with unit and $S \subset R$ a subset. Then the ideal generated by $S$ is an ideal of $R$.
\end{prop}

\subsection{Radical ideal}

\begin{definition}
	Let $R$ be a commutative ring with unit and $I \subset R$ an ideal. We define the \emphdef{radical of $I$} by
	\[
		\rad{I} = \sqrt{I} = 
		\{
			a \in R \mid a^n \in I \text{ for some } n > 0
		\}
	\]
	We say that $I$ is a \emphdef{radical ideal} when $I = \rad{I}$.
\end{definition}

\begin{prop}
	Let $R$ be a commutative ring with unit and $I \subset R$ an ideal.
	\begin{enumerateprop}
		\item The radical $\rad{I}$ is an ideal of $R$.
		\item $I \subset \rad{I} = \rad{\rad{I}}$
	\end{enumerateprop}
\end{prop}

\begin{definition}
	Let $R$ be a commutative ring with unit. We define the \emphdef{nilradical} of $R$ as the radical of the zero ideal, that is,
	\[
		\mathfrak{N}_R = 
		\rad{0} = 
		\{
			a \in R \mid a^n = 0 \text{ for some } n > 0
		\}
	\]
	We say that $R$ is \emphdef{reduced} whenever it has no nilpotent elements different from zero, that is, when $\mathfrak{N}_R = 0$.
\end{definition}

\begin{prop}
	Let $R$ be a commutative ring with unit. The reduction of $R$ is
	\[
		R_\text{red} = \quotient{R}{\mathfrak{N}_R}
	\]
	which is a reduced ring. If $R$ is already a reduced ring, then $R \simeq R_\text{red}$.
\end{prop}

\subsection{Colon and saturation ideals}

\begin{definition}
	Let $R$ be a commutative ring with unit and $I, J \subset R$ ideals of $R$. 
	\begin{enumeratedef}
		\item The \emphdef{colon ideal} of $J$ with respect to $I$ is $(I \, : \, J) = \{ a \in R \mid a J \subset I \}$.
		\item The \emphdef{annihilator of $J$} is $(0 \, : \, J) = \ann{R}{J}$.
		\item The \emphdef{saturation} of $J$ with respect to $I$ is $(I \, : \, J^\infty) = \{ a \in R \mid a J^n \subset I \text{ for some } n > 0 \}$.
	\end{enumeratedef}
\end{definition}

\begin{prop}
	Let $R$ be a commutative ring with unit and $I, J \subset R$ ideals of $R$. Then $(I \, \colon \, J)$, $\ann{R}{J}$ and $(I \, : \, J^\infty)$ are ideals of $R$.
\end{prop}

\subsection{Extension and contraction of ideals}

\begin{definition}
	Let $R, S$ be rings and $f \colon R \rightarrow S$ a ring homomorphism. Let $I \subset R$ and $J \subset S$ be ideals. We define:
	\begin{enumeratedef}
		\item The \emphdef{extension} of $I$ by $I^e = \{ b_1 f(x_1) + \cdots + b_n f(x_n) \mid x_i \in I, \, b_i \in S, \, n > 0 \}$.
		\item The \emphdef{contraction} of $J$ by $J^c = f^{-1}(J) = \{ a \in R \mid f(a) \in J \}$.
	\end{enumeratedef}
\end{definition}

\begin{prop}
	Let $R, S$ be rings and $f \colon R \rightarrow S$ a ring homomorphism. Let $I \subset R$ and $J \subset S$ be ideals. Then,
	\begin{enumerateprop}
		\item The \emphdef{extension} $I^e \subset S$ is an ideal of $S$. Moreover, it is the ideal generated by $f(I) \subset S$.
		\item The \emphdef{contraction} $J^c \subset R$ is an ideal of $R$.
	\end{enumerateprop}
\end{prop}

\begin{prop}
	Let $R, S$ be rings and $f \colon R \rightarrow S$ a ring homomorphism. Let $I \subset R$ and $J \subset S$ be ideals. Then,
	\begin{enumerateprop}
		\item $I \subset I^{e c}$
		\item $I^c = I^{c e c}$
		\item $J \supset J^{c e}$
		\item $J^e = J^{e c e}$
	\end{enumerateprop}
\end{prop}

